%=========================================
% 	Introduction  		 
%=========================================
\chapter{Introduction}

\section{Background}
In manufacturing, the internet of things and other innovative technology trends are redefining the role of Information Technology (IT) in production. Multiple projects aiming at increasing data transparency as well as overall shop flow productivity are carried out year in year out. One of such project is Smart Assistance for Humans in Production (SmARPro). The SmARPro project is funded by the German Federal Minister of Education and Research (BMBF): following the increase of complexity in production operations due to innovative solutions for information driven manufacturing  is a context-sensitive semantic data management platform for retrieving, storing, processing and presentation of data. The goal of SmARPro is to create a virtual twin of existing production and logistics system of a manufacturing plant with a special focus on human machine interaction via Augmented Reality. The Robert Bosch manufacturing plant in Homburg was used as a sample plant and the grinding machine “BAHMÜLLER UTRA twinner” as test machine. The SmARPro system consist of three main components: 
SmARPro Platform -  a semantic data management platform for the context-sensitive storage, processing and provision of data and information. SmARPro platform creates a virtual image of the existing production and logistics system of the Bosch plant in Homburg, which is constantly updated.
SmARPro SmartDevices - are Embedded Systems, which serve the connection of existing (in some cases also old) as well as new machines and plants to the SmARPro Platform. They are attached to the equipment in order to record sensor data (from dedicated sensors or existing control systems) and convert it into semantic models. Thus, any machines and systems can be enabled to Cyber-Physical Production Systems (CPPS).
SmARPro Wearable -  are electronic devices such as tablet PCs or Smart glasses that can be used as augmented reality-based assistance systems. They are used to assist employees in decision-making processes by obtaining, rendering and presenting information from the SmARPro Platform contextually (i.e., collaborator role, location, and task-related) information. They enable the exchange of information and the interaction between the employee and the machine

within a production system that can be integrated into the overall system by linking with the sensors developed in the research project. available, such as data glasses, tablet PCs and smartphones. They serve the exchange of information and the interaction between the employee and the machine. To support a future-oriented use of these main components, innovative methods and procedures for decentralized factory control are being developed.
The term "Augmented Reality" describes the augmented reality that is achieved by adding additional information to the wearables and other aids. This is done, for example, by a smartphone, which records the reality with its camera, and additional information is displayed on the screen at prominent points.

\section{Problem Statement}
In a manufacturing factory such as the Bosch plant in Homburg, production and business processes generate huge amounts of data. These data can be used to increase the productivity of workers as well as machines. But due to lack of communication between workers and their production facilities, most of these data remain inaccessible and thus can’t be put to used. Thanks to today's digitalization with industry 4.0, we can develop a system which creates a digital twin of a physical production plant. This digital twin is the SmARPro System. As mentioned earlier, this system comprises mainly of three key components: smart device, core platform and wearables.  The SmARPro system comes with many challenges. The representation of context-based information on wearables for workers is one of them. This thesis deals with the real time presentation of machine data from the SmARPro platform on wearables for workers in a smart factory. To tackle this issue of presenting information on wearable, we first need to answer the following questions:
\begin{itemize}
\item How can a business process be reproduced and integrated on the SmARPro platform?
\item How can human machine interaction be made possible within a business process chain?
\end{itemize}
	
The goal of this thesis is to answer the above questions. The sub section below provides a solution approach to tackle this challenge.


\section{Solution Approach}
With the rapid development of technology in the recent year plenty of innovation has been brought forward to aid facilitate our lives, one of this innovation is the SmARPro platform. using smart glasses which are connected to the SmARPro platform, machine operators can use augmented reality to analyze and solve complex problems quickly.  the objective of the system is to assist machine operators with task such as machine repair in a scenario of a machine tool break down or mal function. Another task that come up after this machine repair by a machine operator is to provide a report for the machine break down after performing this repair

\section{Outline}

This subsection outlines the thesis structure. The chapters are structured as follows: 
Foundation -	The foundation chapter will give a brief introduction of the industry partner where this thesis was conducted, then moves on to further explain the thesis keywords mentioned earlier in the introduction section namely: smart device, SmARPro platform and wearables. Finally, the chapter explains main concepts behind the SmARPro system, which include semantic web as well as Event-driven process (EPC).   The keywords and main concepts will later serve as foundation work for the proceeding chapters.
Analysis -	The analysis chapter examines and explains sample relevant business use case scenario namely: Machine Tool Repair and Overall Equipment Effectiveness (OEE). It examines various steps of these business processes and pins points key aspect of human machine interaction within the respective process life cycles while seeking to provide the best interaction with the system.
Concept and Design -	This chapter builds on the analysis chapter. It provides sample prototypes of how the use cases mentioned earlier can be modelled and integrated in the SmARPro system while taking into consideration various aspect of Usability and human computer interaction.
Implementation -	In this chapter, we discuss implementations of concrete business processes in a typical manufacturing setting. This section explains how business processes can be modelled and reproduced on the SmARPro platform. Finally, we will demonstrate how this business process from the platform can be presented on wearable devices. To help employees perform complex task such as machine tool repair.
Evaluation -	The evaluation chapter, discusses to what extend our solution approach was successful. We revisit the challenge of communication between machines, humans and their manufacturing plants. which we mentioned earlier in our problem statement and to what extend the work in this thesis could bridge this gap.  

Summary and Future Works -	The final chapter summaries and

