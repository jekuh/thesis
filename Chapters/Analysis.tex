%=========================================
% Related Works
%=========================================
\chapter{Analysis}
Plenty of related research has been conducted and still being made in the domain of Newspaper articles classification. In this section we will have a look at some valuable contributions, which are similar or closely related to our newspaper articles classification problem.  

\section{Document Classification for Newspaper Articles}
In a research paper titled: \textit{Document Classification for Newspaper Articles} based on worked conducted by \cite{Ramdass2009} at Google, they aimed at investigating implementation techniques to automatically classifying articles from the MIT newspaper the Tech, into specific category(Opinion, Sport, News etc), by making use of natural language processing techniques for document classification. This techniques are predicted on the hypothesis that document or article in a specific sections have some distinguish natural language features. They found out salient features for classification ranged from word structure, word frequency, and natural language structure in each document. The data-set at their disposal consisted of a large archive of  already classified documents, making it easy to apply supervised classification techniques.
By randomly dividing their classified achieve into training and test data-set, they conduct experiments using various natural language feature set and statistical techniques, applying different classification algorithms such as Naive Bayes , Maximum Entropy, and comparing the performance in each case.  

